%! Author = itgramic
%! Date = 24.11.23

% Preamble
\chapter{Produktivsystem}
\begin{flushleft}
    Ausserdem sollten ebenfalls zwei Ini-Files präperiert werden.\\
    \textbf{sks0020}\\
    \lstset{style=gra_codestyle}
    \begin{lstlisting}[language=sh, caption=Workstation.ini PROD sks0020,captionpos=b,label={lst:workstation.ini-prod-sks0020},breaklines=true]
    -clean
    -data
    @noDefault
    -configuration c:/temp/phoenix/prod_7_23_1/eclipse/configuration
    -nl
    de_CH
    -vmargs
    -Dorg.eclipse.update.reconcile=false
    -Dphoenix.db=Phoenix
    -Dphoenix.application.workspace.createdir=true
    -Dphoenix.application.workspace=\\\\phoenix\workspace\prod_7_23_1
    -Dphoenix.login.usernamefile=%APPDATA%/Phoenix/login_username_prod
    -Dphoenix.server.nodes=sks0020:8443,sks0020:8543
    -Djavax.net.ssl.trustStore=C:\Program Files (x86)\Phoenix7\truststore.jks
    -Dphoenix.login.enableLDAP=true
    -Dphoenix.application.winlogin=true
    -XX:-CreateMinidumpOnCrash
    \end{lstlisting}
    \textbf{sks0021}\\
    \lstset{style=gra_codestyle}
    \begin{lstlisting}[language=sh, caption=Workstation.ini PROD sks0021,captionpos=b,label={lst:workstation.ini-prod-sks0021},breaklines=true]
    -clean
    -data
    @noDefault
    -configuration c:/temp/phoenix/prod_7_23_1/eclipse/configuration
    -nl
    de_CH
    -vmargs
    -Dorg.eclipse.update.reconcile=false
    -Dphoenix.db=Phoenix
    -Dphoenix.application.workspace.createdir=true
    -Dphoenix.application.workspace=\\\\phoenix\workspace\prod_7_23_1
    -Dphoenix.login.usernamefile=%APPDATA%/Phoenix/login_username_prod
    -Dphoenix.server.nodes=sks0021:8443,sks0021:8543
    -Djavax.net.ssl.trustStore=C:\Program Files (x86)\Phoenix7\truststore.jks
    -Dphoenix.login.enableLDAP=true
    -Dphoenix.application.winlogin=true
    -XX:-CreateMinidumpOnCrash
    \end{lstlisting}
    \begin{mdframed}
    Vorsichht!\\Beim Installieren einer neuen produktiven \texttt{Workstation.exe} wird das ganze Verzeichnis jeweils auf dem Laufwerk \texttt{C:\textbackslash} überschrieben!\\Daher müssen die Inis immer wieder neu erstellt werden.
    \end{mdframed}
\end{flushleft}
\begin{flushleft}
    Folgende Schritte müssen gemacht werden, um den Patch durchzuführen:
    \begin{enumerate}
        \item Snapshost der beiden Server \texttt{sks0020} und \texttt{sks0021} machen
        \item Prüfen ob \texttt{Workstation.exe Prod} auf \texttt{sks0020} und \texttt{sks0021} verbinden kann
        \item Prüfen ob \texttt{Workstation.exe Prod} nur auf \texttt{sks0021} verbinden kann
        \item Den Dienst \texttt{PhoenixJBossEAP\_prod\_3} auf \texttt{sks0020} stoppen
        \item Prüfen ob \texttt{Workstation.exe Prod} nur auf \texttt{sks0020} verbinden kann
        \item Den Dienst \texttt{PhoenixJBossEAP\_prod\_1} auf \texttt{sks0020} stoppen
        \item Server \texttt{sks0020} rebooten
        \item Auf \texttt{sks0020} prüfen, ob keine Patches mehr gezogen werden könnten
        \item Prüfen ob \texttt{Workstation.exe Prod} auf \texttt{sks0020} verbinden kann
        \item Prüfen ob \texttt{Workstation.exe Prod} auf \texttt{sks0021} verbinden kann
        \item Prüfen ob \texttt{Workstation.exe Prod} auf \texttt{sks0020} und \texttt{sks0021} verbinden kann
        \item Den Dienst \texttt{PhoenixJBossEAP\_prod\_4} auf \texttt{sks0021} stoppen
        \item Prüfen ob \texttt{Workstation.exe Prod} nur auf \texttt{sks0021} verbinden kann
        \item Den Dienst \texttt{PhoenixJBossEAP\_prod\_2} auf \texttt{sks0021} stoppen
        \item Server \texttt{sks0021} rebooten
        \item Auf \texttt{sks0021} prüfen, ob keine Patches mehr gezogen werden könnten
        \item Prüfen ob \texttt{Workstation.exe Prod} auf \texttt{sks0021} verbinden kann
        \item Prüfen ob \texttt{Workstation.exe Prod} auf \texttt{sks0020} verbinden kann
        \item Prüfen ob \texttt{Workstation.exe Prod} auf \texttt{sks0020} und \texttt{sks0021} verbinden kann
        \item Nach einem Tag ohne Fehler, die Snapshost der beiden Server \texttt{sks0020} und \texttt{sks0021} löschen
    \end{enumerate}
\end{flushleft}
\begin{flushleft}
    Das ganze noch als Sequenzdiagramm dargestellt:
    \begin{figure}[H]
        \centering
        \includegraphics[width=1\linewidth]{source/prod/sequenzdiagramm_prod}
        \caption{PROD-Sequenzdiagramm}
        \label{fig:prod-sequenzdiagramm}
    \end{figure}
\end{flushleft}